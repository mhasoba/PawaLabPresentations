\documentclass[25pt,margin=18mm, innermargin=12mm]{tikzposter}
\usepackage[utf8]{inputenc}
\geometry{paperwidth=84.1cm,paperheight=118.9cm}
% \usetheme{Simple}
%\usecolorstyle{Spain}
% \usebackgroundstyle{Empty}
% \usetitlestyle{Filled}
%y\useblockstyle{Basic}
\tikzposterlatexaffectionproofoff
\usepackage{amsmath}

\useblockstyle{Slide}
\renewcommand*\familydefault{cabin}
\usepackage[sfdefault,condensed]{cabin}
\usepackage[T1]{fontenc}
\usepackage{fix-cm}
\usepackage{tikz,multicol}

\definecolor{greenone}{RGB}{21,75,52}
\definecolor{greentwo}{RGB}{25,97,66}
\definecolor{greenthree}{RGB}{16,57,39}
\definecolor{mygray}{RGB}{203,203,203}
\definecolorstyle{myColorStyle} {
    \colorlet{colorOne}{greenone}
    \colorlet{colorTwo}{greentwo}
    \colorlet{colorThree}{greenthree}
    \colorlet{colorFour}{mygray}
    }{
    % Background Colors
    \colorlet{backgroundcolor}{colorOne}
    \colorlet{framecolor}{colorOne}
    % Title Colors
    \colorlet{titlefgcolor}{white}
    \colorlet{titlebgcolor}{colorTwo}
    % Block Colors
    \colorlet{blocktitlebgcolor}{colorFour}
    \colorlet{blocktitlefgcolor}{colorTwo}
    \colorlet{blockbodybgcolor}{white}
    \colorlet{blockbodyfgcolor}{black}
    % Innerblock Colors
    \colorlet{innerblocktitlebgcolor}{white}
    \colorlet{innerblocktitlefgcolor}{black}
    \colorlet{innerblockbodybgcolor}{white}
    \colorlet{innerblockbodyfgcolor}{black}
    % Note colors
    \colorlet{notefgcolor}{black}
    \colorlet{notebgcolor}{white}
    \colorlet{notefrcolor}{white}
}

\usecolorstyle{myColorStyle}

\usebackgroundstyle{Rays}
\usepackage{adjustbox}
\usetitlestyle{Empty}

\makeatletter %To make columns scale to custom width 
\setlength{\TP@visibletextwidth}{\textwidth-2\TP@innermargin}
\setlength{\TP@visibletextheight}{\textheight-2\TP@innermargin}
\makeatother

 
%\graphicspath{{images/}}

%%%%%%%%%%%%%%%%%%%%%%%%%%%%%%%%%%%%%%%%%%%%%%%%%%%%%%%%%%%%%%%%%%%%%%%%

\title{\parbox{\linewidth}{\centering \semiHUGE Scaling up individual metabolism to ecosystem fluxes
}}

\author{\vspace{20pt}
  {\begin{minipage}{10em}
    \hfill
    \flushleft
    \includegraphics[trim=0cm 8cm 0cm 8cm,clip,width=10cm]{Imperial_Whitelogo.pdf}
  \end{minipage}
    \vspace{-20pt}
    \hspace{8pt}}
  {\begin{minipage}{18em}
    \centering
        { \LARGE Sofia Sal, Gabriel Yvon-Durocher \& Samraat Pawar \\           
          s.sal-bregua@imperial.ac.uk  }
  \end{minipage}
    \vspace{-20pt}
  \hspace{8pt}}
  {\begin{minipage}{10em}
    \centering
    \hfill
    \centering
    \includegraphics[trim=0cm 8cm 0cm 8cm,clip,width=10cm]{ExeterLogo.pdf}
  \end{minipage}
  \vspace{-20pt}}
  }



    
%%%%%%%%%%%%%%%%%%%%%%%%%%%%%%%%%%%%%%%%%%%%%%%%%%%%%%%%%%%%%%%%%%%%%%%%
\makeatletter
\newcommand\semiHUGE{\@setfontsize\semiHUGE{82}{27.38}}
\makeatother

\begin{document}
\node[xshift=-42.05cm, at=(topright),opacity=0.9]{\includegraphics[width=\paperwidth]{head1low.jpg}}

\maketitle[titletoblockverticalspace=25mm]

  %%%%%%%%%%%%%%%%%%%%%%%%%%%%%%%%%%%%%%%%%%%%%%%%%%%%%%%%%%%%%%%%%%%%%%%%

\block{Do species-level thermal performance curves (TPCs) scale up to ecosystem fluxes?}{

  \begin{minipage}[]{0.45\linewidth}
    \centering
    \includegraphics[trim=6cm 2cm 4cm 6cm,clip,width=34cm]{EcoModelFigure2.pdf}
    
  \end{minipage}   
  \hspace{.8cm}    
  \begin{minipage}[]{0.52\linewidth}

    %% {\bf Scaling up} is a linear mapping (strictly what geometric scaling means) between the lower-level phenomenon (here,individual-level or intra-specific thermal responses) and the emergent one of ecosystem functioning. \\

   Previous studies assume that the temperature dependence of ecosystem function is a simple scaling up of all the component species’ thermal responses. In this case, predicting the effects of climatic warming or cooling on ecosystem function would be a relatively straightforward task. 

   \vspace{-10pt}
   \subsection*{Data}

    We combine new theory and data on the temperature dependence of key metabolic traits (photosynthesis and respiration rates) at both:
    \begin{itemize} \itemsep5pt 
    \item \textit {Species level:} more than 300 different species of terrestrial plants (\textit {Biotraits Database})
    \item \textit{Ecosystem level:} 118 local terrestrial ecosystems across the world (\textit{Fluxnet Database})
    \end {itemize}

    \vspace{-10pt}
    \subsection*{We ask...}
  
    \begin{itemize} \itemsep5pt 
    \item Are differences in species-level temperature-dependence of photosynthesis and respiration reflected in the ecosystem thermal response? 
    \item Do the full unimodal thermal responses of metabolic rate matter for mapping individual TPCs to ecosystem-level fluxes?
    \item Is a simple scaling from species-level TPCs sufficient to predict ecosystem-level responses?
    \end{itemize}\\


        \vspace{10pt}
   We present a preliminary analysis of instraspecific data, to get the patterns necessary for parameterizing a model, and ecosystem flux data for validation.
    \end{minipage}
  
}

\begin{columns}
  %%%%%%%%%%%%%%%%%%%%%%%%%%%%%%%%%%%%%%%%%%%%%%%%%%%%%%%%%%%%%%%%%%%%%%%%
  \column{0.345}
  \block{Parametrization}{
    
    \centering
    \includegraphics[width=\linewidth]{IntraspPlots.pdf}
    \footnotesize Representative set of individual-level plot pairs for R (left) and NP (right). Each pair of R-NP corresponds to the same experiment and species. These species are associated to temperate regions.\\
    
     \vspace{20pt}
    \includegraphics[width=\linewidth]{Terr_EaPairHistInt.pdf}
    \footnotesize Distributions of intraspecific activation energies for R and NP.

  }
  
  \column{0.31}
  \block{Model}{
    As ecosystem flux is essentially dependent upon the difference in biomass production $(P)$ and loss $(R)$ of individuals, a simple equation to map individual metabolism to ecosystem flux on a daily scale is:    

    \begin{align*}
      F & = F_{0} \Big(\sum_{i=1}^{k} {x_i \left(t_d f_{P}(m_i) g_P(T) -
        t_n f_{R}(m_i)g_R(T)\right)}\\[10pt]
      &\quad - \sum_{i=1}^{j} {x_i t f_R(m_i)}g_R(T)\Big) 
    \end{align*}\\

    \vspace{-7pt}
    where:    
    \begin{itemize}\itemsep15pt
     \item $k$ is an autotroph and $j$ is a heterotroph species 
    \item $\sum_{i=1}^{k} x_i$ is the total biomass of the ecosystem 
    \item $m_i$ is species' size
    \item $g(T)$ is contribution of temperature ($T$) dependence
    \item $F_0$ captures sources of variation in ecosystem fluxes that cannot be attributed to either body masses or TPCs
     \item $t_d$ and $t_n$ are the day and night components in a 24-hr cycle for the ecosystem ($t_d + t_n = t$)
    \end{itemize}\\
    \vspace{20pt}
    
    A general form for the distribution of biomasses across species is:
    \begin{align*}
      x_i = h(m_i)
    \end{align*}

    that is, $x_{i}$ is a function $h$ of the body mass of the species.  

  }

  \column{0.345}
  \block{Validation}{ 
    \centering
    \includegraphics[width=\linewidth]{EcosystemPlots.pdf}
    \footnotesize Representative set of ecosystem flux responses to temperature for different sites from tropical and temperate regions. Y axis show respiration in log scale.\\
    
     \vspace{20pt}
    \includegraphics[width=\linewidth]{Terr_EaPairHistEco.pdf}
    \caption{\footnotesize Distributions of activation energies for Reco and NEP at ecosystem level.}
    
  }
\end{columns}


%%%%%%%%%%%%%%%%%%%%%%%%%%%%%%%%%%%%%%%%%%%%%%%%%%%%%%%%%%%%%%%%%%%%%%%%
\begin{columns}

  \column{0.55}
  \block[bodyverticalshift=-9mm]{Summary}{ 	
    \begin{itemize}
      
    \item At the intra-specific level, $E_a$ and $T_{peak}$ for $R$ are usually higher than for $P$.
    \item $T_{peak}$’s are usually much higher than the “characteristic” adaptive environment of the organism, so the full unimodal thermal response doesn't matter for mapping individual TPCs to ecosystem-level fluxes.
     \item Ecosystem responses strongly depend on the temperature fluctuations of each ecosystem regime.
    \end{itemize}         
  }
  %%%%%%%%%%%%%%%%%%%%%%%%%%%%%%%%%%%%%%%%%%%%%%%%%%%%%%%%%%%%%%%%%%%%%%%%
  \column{0.45}
  \block{Ongoing}{
    
    \begin{itemize}\itemsep10pt
    \item What role might acclimation of intraspecific TPCs play on the ecosystem response?
    \item What is an appropriate distribution for species-level biomass abundances $x_i$?
    \item  Which is the effect of non-linear interactions between species? 
    \end{itemize}      
  }
  
\end{columns}

\block[bodyverticalshift=-8mm]{}{
  \footnotesize   
  \begin{itemize}
  \item Enquist, B. J., Economo, E. P., Huxman, T. E., Allen, A. P., Ignace, D. D. and Gillooly, J. F. 2003. Scaling metabolism from organisms to ecosystems. – Nature 423(6940): 639–42.
  \item Savage, V. M. 2004. Improved approximations to scaling relationships for species, populations, and ecosystems across latitudinal and elevational gradients. – J. Theor. Biol. 227(4): 525–534.
  \item Yvon-Durocher, G. and Allen, A. P. 2012. Linking community size structure and ecosystem functioning using metabolic theory. – Philos. Trans. R. Soc. Lond. B. Biol. Sci. 367(1605): 2998–3007.
  \item Yvon-Durocher, et al. 2012. Reconciling the temperature dependence of respiration across timescales and ecosystem types. – Nature 487(7408): 472–6.
  \end{itemize}
  \textit{Acknowledge: This work used eddy covariance data acquired by the FLUXNET community (http://fluxnet.fluxdata.org).}
}


\end{document}
